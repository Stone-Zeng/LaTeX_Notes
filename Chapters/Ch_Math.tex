\begin{myQA}{如何将实部、虚部用 Re 和 Im 表示?}
	在\LaTeXTeX 中,命令 \verb|\Re| 和 \verb|\Im| 得到的是大写哥特体字母
	$\Re$ \indexcmd{\backslash Re} 和 $\Im$ \indexcmd{\backslash Im}。
	这是高德纳在 \filename{plain.tex} 中定义的。
	
	用下面的命令可以将它们重定义为更常用的格式 $\operatorname{Re}$
	和 $\operatorname{Im}$:
\begin{myCode}
\renewcommand{\Re}{\operatorname{Re}}
\renewcommand{\Im}{\operatorname{Im}}
\end{myCode}
	
	使用 \pkg{physics} \indexpkg{physics} 宏包中的 \verb|\Re| 和 \verb|\Im| 也
	可以达到同样的目的。
	这个宏包还把原来的哥特体保存在了命令 \verb|\real| 和 \verb|\imaginary| 中。
	\indexcmd{\backslash real} \indexcmd{\backslash imaginary}
	
	\myRef{\cite{PKG2012physics}}
\end{myQA}

\begin{myQA}{怎样正确使用微分符号?}
	按照传统,$\dd{x}$、$\dd{\theta}$ 等微元的前后均需要留出一个细空格
	(thin space)的距离。在 \LaTeX 中,细空格用命令 \verb|\,| 表示,
	默认情况下相当于 \SI{3}{mu} 的长度。这个 \si{mu} 表示数学单位,
	一个 \si{mu} 等于 $1/18$ 个 \si{em} 的长度。至于 \si{em},假设你是知道的。
	
	下面是一个例子:
\begin{myExampleV}
{
	\begin{alignat*}{2}
		&\text{错误的:} \quad \iint f(x,y) \mathrm{d}x \mathrm{d}y
		&\quad\quad& \iiint \mathrm{d}r \mathrm{d}\varphi \mathrm{d}\theta \\
		&\text{正确的:} \quad \iint f(x,\,y) \, \mathrm{d}x \, \mathrm{d}y
		&& \iiint \mathrm{d}r \, \mathrm{d}\varphi \, \mathrm{d}\theta
	\end{alignat*}
}
\begin{alignat*}{2}
(|\tab|)&\text{错误的:} \quad
(|\tab|)\iint f(x,y) \mathrm{d}x \mathrm{d}y
(|\tab|)&\quad \quad&
(|\tab|)\iiint \mathrm{d}r \mathrm{d}\varphi \mathrm{d}\theta \\
(|\tab|)&\text{正确的:} \quad
(|\tab|)\iint f(x,\,y) \, \mathrm{d}x \, \mathrm{d}y
(|\tab|)&&
(|\tab|)\iiint \mathrm{d}r \, \mathrm{d}\varphi \, \mathrm{d}\theta
\end{alignat*}
\end{myExampleV}
	但是“\verb|\,|”一多,写起来会很麻烦,而且容易错。惯例,可以新搞一个定义:
\begin{myExampleH}[0.25]
{
	\newcommand{\dif}{\operatorname{d \!}}
	\begin{gather*}
		\iint f(x, \, y) \dif x \dif y \\
		\frac{\dif f}{\dif x} = \sin x \\
		\frac{\dif{^2 g}}{\dif{x^2}} = \cos x
	\end{gather*}
}
%% ----------导言区---------- %%
\DeclareMathOperator{\dif}{d \!}
% 或者 \newcommand{\dif}{\operatorname{d \!}}
%% ----------导言区---------- %%
\begin{gather*}
(|\tab|)\iint f(x, \, y) \dif x \dif y \\
(|\tab|)\frac{\dif f}{\dif x} = \sin x \\
(|\tab|)\frac{\dif{^2 g}}{\dif{x^2}} = \cos x
\end{gather*}
\end{myExampleH}
	这里有两种实现方法。如果用 \verb|\DeclareMathOperator| 的话,
	要注意该声明的代码只能放在导言区。另外这个命令需要
	\pkg{amsmath} \indexpkg{amsmath} 宏包的支持。
	
	这两种方法的原理是类似的。利用数学算子
	(\verb|\DeclareMathOperator| 或 \verb|\operatorname|)
	的命令保证了微分算子前面的间距,又用“\verb|\!|”去掉了它后面的间距。
	同时,这个命令还自动把“$\mathrm{d}$”切换到了直立字体(\verb|\mathrm|)。
	
	如果微分算子“$\mathrm{d}$”后面跟的不是字母,那就不要偷懒,
	该加的分组括号不能少,要不然 bug 会来得出人意料:
\begin{myExampleV}
{
	\newcommand{\dif}{\operatorname{d \!}}
	\begin{alignat*}{4}
		&\text{不好的:} &\quad&
			\dif (\cos x) &\quad \quad&
			\dif \left(\cos x\right) &\quad \quad&
			\dif \left(\frac{\ln x}{x}\right) \\
		&\text{好的:} &&
			\dif{(\cos x)} &&
			\dif{\left(\cos x\right)} &&
			\dif{\left(\frac{\ln x}{x}\right)}
	\end{alignat*}
}
\newcommand{\dif}{\operatorname{d \!}}

\begin{alignat*}{4}
(|\tab|)&\text{不好的:} &\quad&
(|\tab\tab|)\dif (\cos x) &\quad \quad&
(|\tab\tab|)\dif \left(\cos x\right) &\quad \quad&
(|\tab\tab|)\dif \left(\frac{\ln x}{x}\right) \\
(|\tab|)&\text{好的:} &&
(|\tab\tab|)\dif{(\cos x)} &&
(|\tab\tab|)\dif{\left(\cos x\right)} &&
(|\tab\tab|)\dif{\left(\frac{\ln x}{x}\right)}
\end{alignat*}
\end{myExampleV}
	
	普通括号跟在微分算子“$\mathrm{d}$”后面,间距太小,不好看。
	但是如果是定界符括号,间距却又是正常的。所以还是老老实实加上“\verb|{}|”。
	
	\blankline
	
	根据 ISO 80000-2 的要求,微分算子应使用直立的“$\mathrm{d}$”。
	但是,如果你就是任性,非要用斜体的“$d$”,也是可以的。
	把之前代码中的 \verb|d| 用 \verb|\mathnormal{d}| 来代替,
	就可以强制使用斜体(当然前提要求默认数学字体就是倾斜的)。
	
	\blankline
	
	\pkg{physics} \indexpkg{physics} 宏包中定义了 \verb|\differential| 命令
	(简写为 \verb|\dd|),它涵盖了之前我们做的事情,又通过可选参数引入了上标。
	对于圆括号,它还给出了自动处理的解决方案:
\begin{myExampleH}
{
	\begin{equation*}
		\dd x \quad \dd{\theta} \quad \dd[2]{x} \quad
		\dd(\sin \theta) \quad \dd[2](\frac{\ln x}{x})
	\end{equation*}
}
%% ----------导言区---------- %%
\usepackage{physics}
%% ----------导言区---------- %%
\begin{equation*}
(|\tab|)\dd x \quad
(|\tab|)\dd{\theta} \quad
(|\tab|)\dd[2]{x} \quad
(|\tab|)\dd(\sin \theta) \quad
(|\tab|)\dd[2](\frac{\ln x}{x})
\end{equation*}
\end{myExampleH}
	
	另外,这个宏包提供的 \verb|\derivative| 命令(简写为 \verb|\dv|)
	可以类似的手法处理导数:
\begin{myExampleH}
{
\begin{equation*}
\dv{x} \quad \dv{R}{\theta} \quad \dv[n]{f}{x} \quad
\dv{r}(\frac{\ln r}{r})
\end{equation*}
}
%% ----------导言区---------- %%
\usepackage{physics}
%% ----------导言区---------- %%
\begin{equation*}
(|\tab|)\dv{x} \quad
(|\tab|)\dv{R}{\theta} \quad
(|\tab|)\dv[n]{f}{x} \quad
(|\tab|)\dv{r}(\frac{\ln r}{r})
\end{equation*}
\end{myExampleH}
	
	@brian-ammon
	
	\myRef{\citet{PKG2012physics}}
\end{myQA}
