\begin{myQA}{如何将实部、虚部用 Re 和 Im 表示?}
	在\LaTeXTeX 中,命令 \verb|\Re| 和 \verb|\Im| 得到的是大写哥特体字母
	$\Re$ \indexcmd{\backslash Re} 和 $\Im$ \indexcmd{\backslash Im}。
	这是高德纳在 \filename{plain.tex} 中定义的。
	
	用下面的命令可以将它们重定义为更常用的格式 $\operatorname{Re}$
	和 $\operatorname{Im}$:
\begin{myCode}
\renewcommand{\Re}{\operatorname{Re}}
\renewcommand{\Im}{\operatorname{Im}}
\end{myCode}
	
	使用 \pkg{physics} \indexpkg{physics} 宏包中的 \verb|\Re| 和 \verb|\Im| 也
	可以达到同样的目的。
	这个宏包还把原来的哥特体保存在了命令 \verb|\real| 和 \verb|\imaginary| 中。
	\indexcmd{\backslash real} \indexcmd{\backslash imaginary}
	
	\myRef{115,116}
\end{myQA}

\begin{myQA}{怎样正确使用微分符号?}
	按照传统,$\dd{x}$、$\dd{\theta}$ 等微元的前后均需要留出一个细空格(thin space)的距离。在 \LaTeX 中,细空格用命令 \verb|\,| 表示,默认情况下相当于 \SI{3}{mu} 的长度。这个 \si{mu} 表示数学单位,一个 \si{mu} 等于 $1/18$ 个 \si{em} 的长度。至于 \si{em},假设你是知道的。
	
	下面是一个例子:
\begin{myExampleV}
{
	\begin{alignat*}{2}
		&\text{错误的:} \quad \iint f(x,y) \mathrm{d}x \mathrm{d}y
		&\quad\quad& \iiint \mathrm{d}r \mathrm{d}\varphi \mathrm{d}\theta \\
		&\text{正确的:} \quad \iint f(x,\,y) \, \mathrm{d}x \, \mathrm{d}y
		&& \iiint \mathrm{d}r \, \mathrm{d}\varphi \, \mathrm{d}\theta
	\end{alignat*}
}
\begin{alignat*}
(|\tab|)&\text{错误的:} \quad
(|\tab|)\iint f(x,y) \mathrm{d}x \mathrm{d}y
(|\tab|)&\quad \quad&
(|\tab|)\iiint \mathrm{d}r \mathrm{d}\varphi \mathrm{d}\theta \\
(|\tab|)&\text{正确的:} \quad
(|\tab|)\iint f(x,\,y) \, \mathrm{d}x \, \mathrm{d}y
(|\tab|)&&
(|\tab|)\iiint \mathrm{d}r \, \mathrm{d}\varphi \, \mathrm{d}\theta
\end{alignat*}
\end{myExampleV}
但是“\verb|\,|”一多,写起来会很麻烦,而且容易错。惯例,可以新搞一个定义:
\begin{myExampleH}[0.25]
{
	\newcommand{\dif}{\operatorname{d\!}}
	\begin{gather*}
		\iint f(x, \, y) \dif x \dif y \\
		\frac{\dif f}{\dif x} = \sin x \\
		\frac{\dif{^2 g}}{\dif{x^2}} = \cos x
	\end{gather*}
}
%% ----------导言区---------- %%
\DeclareMathOperator{\dif}{d \!}
% 或者 \newcommand{\dif}{\operatorname{d \!}}
%% ----------导言区---------- %%
\begin{gather*}
(|\tab|)\iint f(x, \, y) \dif x \dif y \\
(|\tab|)\frac{\dif f}{\dif x} = \sin x \\
(|\tab|)\frac{\dif{^2 g}}{\dif{x^2}} = \cos x
\end{gather*}
\end{myExampleH}
未完待续……
\end{myQA}
