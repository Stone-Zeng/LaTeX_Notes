\begin{myQA}{如何优雅地在科技文献中使用句点?}
	在一般中文文章中,通常使用句号“{\CJKfamily{宋体}。}”。在 Unicode 中,
	它的代码为 \verb|U+3002|。而在科技文献中,经常使用句点“。”(即全角
	英文句号),它的 Unicode 代码为 \verb|U+FF0E|。类似符号还有:半角英文句号
	\verb|U+002E|“.”,半角中文句号 \verb|U+FF61|
	\footnote{本文所用字体并不包含该符号,可以改用微软雅黑等字体来显示。}
	等。这篇文章可以假装是科技文献,因此用了句点。
	
	我们有三种使用句点的方案。
	
	\emph{普青方案}:使用 \pkg{xeCJK} 宏包的朋友,在调用字体时使用
	\verb|Mapping = fullwidth-stop| 选项,就可以将正常句号转换成句点。选项
	\verb|Mapping = full-stop| 的作用恰恰相反。
	
	\emph{文青方案}:在导言区添加如下代码即可。此代码第一行将
	“{\CJKfamily{宋体}。}”设置为活动符,并将其定义为命令,从而输出一个句点。
\begin{myCode}[|]
\catcode`\|{\CJKfamily{宋体}。}| = \active
\newcommand{|{\CJKfamily{宋体}。}|}{.}
\end{myCode}
	
	\emph{二青方案}:利用编辑器全文替换。不解释。
	
	\myRef{115,116}
\end{myQA}

\begin{myQA}{这是二}
	vsDVD字符v的v倍数个纳税人ecb
\end{myQA}

\begin{myQA}{这是三}
	vsDVD字符v的vecb案范围大气的
\end{myQA}