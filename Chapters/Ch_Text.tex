\begin{myQA}{如何优雅地在科技文献中使用句点?}
	在一般中文文章中,通常使用句号“{\CJKfamily{宋体}。}”。在 Unicode 中,
	它的代码为 \verb|U+3002|。而在科技文献中,为了与一些下标区分,
	经常使用句点“.”(即全角英文句号)
	\footnote{在新的《标点符号用法:GB/T 15834—2011》
		\textsuperscript{\cite{GB/T15834-2011标点符号}} 中,
		这一条被删去了。是否采纳,您看着办。}
	,它的 Unicode 代码为 \verb|U+FF0E|。
	类似符号还有:半角英文句号达到 \verb|U+002E|“.”,
	半角中文句号 \verb|U+FF61|
	\footnote{本文所用字体并不包含该符号,可以脑补一下,
		或者改用微软雅黑等字体来显示。}
	等。这篇文章可以假装是科技文献,因此用了句点。
	
	我们有三种使用句点的方案。
	
	\emph{普青方案}:使用 \pkg{xeCJK} \indexpkg{xeCJK} 宏包的朋友,
	在调用字体时使用
	\verb|Mapping = fullwidth-stop| \indexcmd{Mapping} 选项,
	就可以将正常句号转换成句点。选项 \verb|Mapping = full-stop|
	的作用恰恰相反。
	
	\emph{文青方案}:在导言区添加如下代码即可。此代码第一行将
	“{\CJKfamily{宋体}。}”设置为活动符,并将其定义为命令,从而输出一个句点。
\begin{myCode}
\catcode`\(|{\CJKfamily{宋体}。}|) = \active
\newcommand{(|{\CJKfamily{宋体}。}|)}{.}
\end{myCode}
	\indexcmd{\backslash catcode} \indexcmd{\backslash active}
	
	\emph{二青方案}:利用编辑器全文替换。缺点是这种方式并不优雅。
	
	\myRef{\citet{PKG2016xeCJK}}
\end{myQA}

%HACK:20160708 PDF 标签显示仍为句号,这里改用句点
\begin{myQA}{各种连字符、破折号的区别.}
	在英文中主要有三种:hyphen、en dash 和 em dash。
	
	\begin{myItemize}
		\item  hyphen,-,即连字符,Unicode 代码为 \verb|U+002D|
			\footnote{严格来说,\verb|U+002D| 表示“连字符或减号”。
				Unicode 还单独定义了一个连字符 \verb|U+2010|,
				但是输入不太方便(\verb|\symbol{8208}|)。}
			,用普通键盘就可以直接输入。
		\item en dash,--,即字母“n”宽度的破折号,
			Unicode 代码为 \verb|U+2013|。
			在\LaTeXTeX 中,输入 \verb|--| (即两个 hyphen)可以通过预先
			设定的连字功能得到 en dash。
		\item em dash ,---,想必就是字母“m”宽度的破折号,
			Unicode 代码为 \verb|U+2014|。
			可以通过输入 \verb|---| (即三个 hyphen)来得到。
	\end{myItemize}
	
	\blankline
	
	中文中的连字符也有三种:短横线“{\Songti -}”、一字线“—”和浪纹线“~”。短横线就是上文的 hyphen,而一字线则是 em dash\footnote{观察仔细的话,可以发现它们其实略有差异。这是由于上文的 hyphen 和 em dash 使用了英文字体,而这里的短横线和一字线使用了中文字体。}。剩下的浪纹线,Unicode 代码为 \verb|U+FF5E|。它的输入比较麻烦,毕竟\LaTeXTeX 不是原生支持 Unicode 的。较安全的方法是用 \verb|\symbol{65374}| \indexcmd{\backslash symbol}输入。如果使用 \hologo{XeTeX} 编译,直接在源代码中输入该符号也可以得到。
	
	中文中的破折号是“——”,实际上就是两个连在一起的 em dash。使用中文输入法,一般很容易输入。顺便一说,之前的一字线,可以由破折号删掉一半得到。这实际上是最简单的方法。
\end{myQA}

\begin{myQA}{如何在正文中使用上下标?}
	上下标的命令分别是 \verb|\textsuperscript|
	和 \verb|\textsubscript|
	\footnote{在早期版本的 \LaTeX 中,
		使用 \verb|\textsubscript| 命令可能需要调用
		\pkg{fixltx2e} \indexpkg{fixltx2e} 宏包。}
	\indexcmd{\backslash textsuperscript}
	\indexcmd{\backslash textsubscript}。注意要加分组括号,
	否则这两个命令只会对其后的第一个字符起作用。
	
	利用行内公式,也可以实现上下标:
\begin{myExampleV}
{
	\vspace{1 ex}
	
	这是第一个 $^\text{上标}$ 啦啦啦。
	This is another $^\text{supersciprt}$ lalala.
	
	这是第三个$^\text{上标}$啦啦啦。
	This is the fourth$^\text{supersciprt}$lalala.
	
	这是一个$_\text{下标}$。
	This is a $_\text{subscript}$.
}
%% 不要忘记 \usepackage{amsmath}
这是第一个(|\textvisiblespace|)$^\text{上标}$(|\textvisiblespace|)啦啦啦。
This is another(|\textvisiblespace|)$^\text{supersciprt}$(|\textvisiblespace|)lalala.

这是第三个$^\text{上标}$啦啦啦。
This is the fourth$^\text{supersciprt}$lalala.

这是一个$_\text{下标}$。
This is a(|\textvisiblespace|)$_\text{subscript}$.
\end{myExampleV}
	在中文环境中,上下标前后的空格会自动加上,
	无论是不是手动添加。有的时候这会造成麻烦。
	而且这种手段略显猥琐,不推荐使用。
	
	\myRef{\citet{TSE2012superscript,TSE2010subscript}}
\end{myQA}