\begin{myQA}{\texttt{\textbackslash newcommand} 和
	\texttt{\textbackslash newcommand*} 有什么区别?}
\indexcmd{\backslash newcommand} \indexcmd{\backslash newcommand*}
%
首先来介绍一点人生经验。

……

用 \code|\newcommand| 来定义命令时,可以使用“长参数”,
即参数中可以带空行或者 \code|\par| 命令;
而用 \code!\newcommand*! 来定义命令时,则不可以使用“长参数”。
%HACK:20161002 这里用 \code! ! 是为了防止与逃脱字符 *| 冲突

\myRef{\citet{TSE2012newcommand}}
\end{myQA}